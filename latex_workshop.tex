\documentclass{beamer}
\usetheme{AnnArbor}
\usepackage[utf8]{inputenc}
\usepackage[spanish]{babel}
% \usepackage{verbatim}
% % \usecolortheme{seagull}
\setbeamertemplate{navigation symbols}{}

\newcommand{\myframe}[2]
    {
    \begin{frame}{#1}
        #2
    \end{frame}
    }

\newcommand{\myblock}[2]
    {
    \begin{block}{#1}
        #2
    \end{block}
    }

\newcommand{\oneColumn}[4]
    {
    \begin{columns}[c]
        \begin{column}{#1\textwidth}
        \end{column}
        \begin{column}{#2\textwidth}
             #4
        \end{column}
        \begin{column}{#3\textwidth}
        \end{column}
    \end{columns}
    }

\newcommand{\twoColumns}[4]
    {
    \begin{columns}[c]
        \begin{column}{#1\textwidth}
            #3
        \end{column}
        \begin{column}{#2\textwidth}
             #4
        \end{column}        
    \end{columns}
    }


\newcommand{\myfigure}[1]
  {
  \begin{figure}
    \center
    \input{#1}
  \end{figure}
  }

\title{Taller de Creación de Documentos con \LaTeX}

\begin{document}

	\maketitle

	\myframe{}{
	\center
	\Huge Presentaciones

	}

\section{Instalación}

	\myframe{Instalación}{
		\begin{block}{Compilador (Distribución)}
			\begin{description}
			\item[Windows] Miktex\\
			http://miktex.org/download

			\item[Mac] Mactex\\
			https://tug.org/mactex/
			\end{description}
		\end{block}

		\begin{block}{Editor}
			TexStudio \\
			http://sourceforge.net/projects/texstudio/ \\
			(Windows y Mac)
		\end{block}
	}

	\myframe{Instalación}{
		\begin{block}{Lector PDF}
			\begin{description}
			\item[Windows] SumatraPDF \\
			http://www.sumatrapdfreader.org/free-pdf-reader.html

			\item[Mac] Skim \\
			http://skim-app.sourceforge.net/
			\end{description}
		\end{block}
	}


\section{Introducción}

	\myframe{Qué es \LaTeX?}{

		\begin{block}{De Wikipedia}
		``\LaTeX~ es un sistema de composición de textos, orientado a la creación de documentos escritos que presenten una alta calidad tipográfica'' \\
		\end{block}

		\begin{block}{}
			\itemize
			\item Creado por Leslie Lamport
			\item Basado en \TeX (creado por Donald Knuth)
		\end{block}

	}

	\myframe{Características de \LaTeX}{

		This is texto
	}

	\myframe{Ventajas y Desventajas de \LaTeX}{
		Las ventajas son.
	}


	\begin{frame}{Usos de \LaTeX}

	\end{frame}

	\begin{frame}[fragile]{Elementos de un documento}
		\begin{columns}[t]
  	\begin{column}{0.45\textwidth}
  	\myblock{Elementos}{
    	\begin{itemize}
				\item Preámbulo
				\begin{itemize}
					\item Clase
					\item Paquetes a usar
					\item Especificaciones
				\end{itemize}
				\item Cuerpo
				\begin{itemize}
					\item Resumen
					\item Secciones
					\item Bibliografía

				\end{itemize}
			\end{itemize}
		}
    \end{column}
    \begin{column}{0.45\textwidth}
    \begin{block}{Estructura}
      \begin{verbatim}
				\documentclass{...}
				...
				\begin{document}
				...
				\end{document}
			\end{verbatim}
		\end{block}
    \end{column}
    \end{columns}
	\end{frame}


	\begin{frame}[fragile]{Ejercicio: Primer documento en TexStudio}

		\myblock{Crear}{
		Archivo $\rightarrow$ Nuevo
		}

		\begin{block}{Escribir}
		 \begin{verbatim}
			\documentclass{article}
			\begin{document}
			My primer documento
			\end{document}
		 \end{verbatim}
		\end{block}

		\myblock{Compilar y Ver}{
		Presionar la doble flecha verde, o presionar F5.
		}

		\myblock{Guardar}{
		Control + S $\rightarrow$ ``primerdocumento.tex''
		}

	\end{frame}

	\begin{frame}[fragile]{Ejercicio: Primer documento en TexStudio}
		\begin{block}{Estilizar}
			Agregar nombre y título del documento en preámbulo
			\verb|\name{Juan Perez}| \\
			\verb|\title{Documento en \LaTeX}| \\

		\end{block}

		\begin{block}{Compilar, F5}
			Nada va a pasar
		\end{block}

		\begin{block}{Arreglo}
			Agregar debajo de \verb|\begin{document}| : \\
			\verb|\maketitle|
		\end{block}
		\myblock{Compilar, F5}{
		}
	\end{frame}

	\section{Conceptos Preliminares}

	\myframe{Comandos y Entornos}{
		\myblock{Comandos}{
		\begin{itemize}
		\item Órdenes que realizan una acción sencilla
		\item Siempre empiezan con \textbackslash
		\item Sintaxis: \textbackslash comando[opciones]\{argumentos\}
		\end{itemize}


		Ejemplos
		\begin{itemize}
		\item \textbackslash usepackage[spanish]\{babel\}
		\item \textbackslash title\{Primer documento\}
		\end{itemize}
		}
	}


	\myframe{Comandos y Entornos}{
		\myblock{Entornos}{
			\begin{itemize}
				\item Órdenes que realizan una acción compleja
				\item Siempre empiezan con \textbackslash begin y se cierran con \textbackslash end
				\item Sintaxis: \\
					\textbackslash begin\{entorno\} \\
					... \\
					\textbackslash end\{entorno\}
			\end{itemize}

			Ejemplos
			\begin{itemize}
				\item \textbackslash begin\{center\} ... \textbackslash end\{center\}
				\item \textbackslash begin\{equation\} ... \textbackslash end\{equation\}
				\item \textbackslash begin\{section\} ... \textbackslash end\{section\}
			\end{itemize}

		}
	}




	\section{Redacción básica}






\end{document}